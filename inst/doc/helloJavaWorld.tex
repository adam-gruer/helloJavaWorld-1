\documentclass[a4paper, 11pt]{article}
\usepackage[OT1]{fontenc}
\usepackage{url}
\usepackage{Sweave}
\begin{document}

% \VignetteIndexEntry{helloJavaWorld}

\title{Hello Java World! A Tutorial for Interfacing to\\ Java Archives
       inside R Packages.}
\author{Tobias Verbeke}
\date{2008-04-28}

\maketitle

\tableofcontents

\section{Introduction}

This document provides detailed guidance on interfacing R to Java archives
inside an R package. The package we will create in this tutorial, the 
\texttt{helloJavaWorld} package, will invoke a very simple Java class, 
the \texttt{HelloJavaWorld} class, from inside an R function of the package, 
the \texttt{helloJavaWorld} function. The objective is to help other people
to make available Java algorithms to the R world, be it to compare results,
or for their own sake.

\section{Structure of the package}

We create a folder, \texttt{helloJavaWorld}, which will be the root folder
of our \texttt{helloJavaWorld} package. For detailed information on R packages
and their structure, the reader is referred to the Writing R Extensions manual. 
This section only highlights elements that are relevant to our situation.
Before detailing the contents of the individual files and folders, we 
provide a summary overview in Figure \ref{fig:pkgContents}.
\begin{figure}
\begin{verbatim}
  helloJavaWorld
      `- inst
          `- doc
              `- helloJavaWorld.Rnw       (this document)
          `- java
              `- hellojavaworld.jar    
      `- man
          `- helloJavaWorld.Rd
      `- R
          `- helloJavaWorld.R
          `- onLoad.R
      `- DESCRIPTION
      `- NAMESPACE
\end{verbatim}
\caption{Overview of package contents for the \texttt{helloJavaWorld} package.}
\label{fig:pkgContents}
\end{figure}


\paragraph{\texttt{DESCRIPTION}}

Inside this folder, we create \texttt{DESCRIPTION} file, 
which can be considered to be the `identity card'
of the package. The most important changes when comparing to 
regular \texttt{DESCRIPTION} files are that 
\begin{itemize}
  \item there must be a package dependency on the
     \texttt{rJava} package (\texttt{Depends} field) as well as 
  \item a system dependency on Java (\texttt{SystemRequirements} field). 
\end{itemize}

\paragraph{\texttt{inst/java}}

We create the folder \verb|inst/java| to host our JAR file. This JAR file
contains a single \texttt{HelloJavaWorld.class} file, generated from the following
\texttt{HelloJavaWorld.java} file.

\begin{verbatim}
  public class HelloJavaWorld {
   
    public String sayHello() {
      String result = new String("Hello Java World!");
      return result;
    }
  
    public static void main(String[] args) {
    }
  } 
\end{verbatim}

As one can see, this is not the typical example of a Hello World application.
The string \texttt{Hello Java World!} is namely not printed to the console,
but returned by the \texttt{sayHello} method. The reason to deviate from this
tradition is that in practice the interfacing to Java classes will nearly 
always result in return values being returned to R by the methods of the
classes in the JAR file(s). 
 

\paragraph{\texttt{R/}}

Two functions are contained in the \texttt{R/} subfolder of the package.
The first function is the function that will assure that the
Java code is made available. The second function will be the
R wrapper to execute the Java HelloWorld class.

Namespaces are recommended in R packages, so we will only
detail how to include the first function when the package
has a namespace. We include a file \texttt{onLoad.R} with
the following content~:

\begin{verbatim}
  .onLoad <- function(libname, pkgname) {
    .jpackage(pkgname)
  }
\end{verbatim}

The \texttt{.onLoad} function is a hook function that
will be run immediately after loading the package.
The function \texttt{.jpackage} that is called inside
the \texttt{.onLoad} function takes care to initialize the JVM
and to add the \texttt{java/} folder of the package to the class path.
Note that when building the package the \verb|inst| level is `taken out', 
so that in the built package the \verb|java/| subfolder will be directly
in the root folder.

The second function we included in the \texttt{R/} folder is responsible
for making the Java class available to the user and is a simple 
wrapper:

\begin{verbatim}
  helloJavaWorld <- function(){
    hjw <- .jnew("HelloJavaWorld")     # create instance of HelloJavaWorld class
    out <- .jcall(hjw, "S", "sayHello")  # invoke sayHello method
    return(out)
  }
\end{verbatim}

The \texttt{.j*} functions such as \texttt{.jnew} and \texttt{.jcall}
are part of the \texttt{rJava} package and documented in their respective
documentation files\footnote{The \texttt{rJava} package can be obtained from 
CRAN; more information on the package can be found at the package home page 
\url{http://www.rforge.net/rJava/}}. For convenience, we provide some minimal
explanation here as well. 

The \texttt{.jnew} function creates a new instance
of a Java object. The first argument (here \texttt{"HelloJavaWorld"}) 
indicates the class of the object to be created. If other arguments are needed 
to construct the Java object, these can be passed as R arguments to the \texttt{.jnew} 
function as well. In this very simple case, however, no supplementary arguments are required 
to create a \texttt{"HelloJavaWorld"} object. By assigning the result of \texttt{.jnew} to
a variable, we obtain a reference to the object (here \texttt{hjw}) that we can use to
further work with the object, or to call Java methods on the newly created object.

The \texttt{.jcall} function allows to call a method (\texttt{"sayHello"}) on a Java object.
The first argument will be a reference to the object, which we stored in the \texttt{hjw} variable
in the previous statement. The second argument contains a character indicating the return type
of the method we call. In this example, the return type will be a String, which has been mapped
to the \texttt{"S"} character. An overview of all return types is offered in Table 
\ref{tab:returntypes}. 

It is important to note that the \texttt{out} object is
no longer a reference to the resulting Java object, but has been evaluated directly
to a string, i.e. an R character vector of length one. The explanation is that the
\texttt{.jcall} method has an argument \texttt{evalString} that is set to 
\texttt{TRUE} by default. If we would have wanted to obtain the reference to
the Java object to further work with it, we would have explicitly 
set \texttt{evalString} to \texttt{FALSE} as in 

\begin{verbatim}
  outRef <- .jcall(hjw, "S", "sayHello", evalString = FALSE) 
\end{verbatim}

To explicitly obtain a string from a Java object reference,
we could then use the function{.jstrVal}, as in
\begin{verbatim}
  .jstrVal(outRef)
\end{verbatim}

The string returned would, of course, be exactly the same
\begin{verbatim}
  "Hello Java World!"
\end{verbatim}

\begin{table}
  \begin{tabular}{lp{8cm}}
  \hline
  Abbreviation   &  Type \\
  \hline
    \texttt{"V"} & void \\
    \texttt{"I"} & integer \\
    \texttt{"D"} & double (numeric) \\  
    \texttt{"J"} & long \\
    \texttt{"F"} & float \\
    \texttt{"Z"} & boolean \\
    \texttt{"C"} & char (integer)\\
    \texttt{"S"} & String \\
    \texttt{"B"} & byte (raw)\\
    \texttt{"L<class>"} & Java object of class \texttt{<class>} 
                       (e.g. \texttt{"Ljava/lang/Object"})\\
    \texttt{"[<type>"}  & Array of objects of type \texttt{<type>} 
                       (e.g. \texttt{"[D"} for an array of doubles)\\
  \hline
  \end{tabular}
  \caption{Overview of the abbreviations that can be used as a shortcut to indicate the
           return type of a Java method in the \texttt{.jcall} function of
           the \texttt{rJava} package.}
  \label{tab:returntypes}
\end{table} 

\paragraph{NAMESPACE}

In the R functions of our package, some functions were used from the
\texttt{rJava} package. As this package has a namespace as well, it
is necessary to import the relevant functions or the whole package into
the package namespace to ensure that the \texttt{rJava} package will be 
loaded as well. Importing the whole package can be done using the 
\texttt{import} directive. 

If, on the other hand, we want to make our functions (or other objects)
available to the package user, we need to export these objects using the
\texttt{export} directive. We will of course not export the \verb|.onLoad| 
function as it is not meant to be used by the package user. The
only function exported, therefore, is the \texttt{helloJavaWorld}
function. The \texttt{NAMESPACE} file will thus have the 
following contents\footnote{For more information on package namespaces, 
please consult Section 1.6 from the Writing R Extensions manual.}~:

\begin{verbatim}
  import("rJava")
  export("helloJavaWorld")
\end{verbatim}

\section{Use of the package}

Once the package is checked, built and installed, we
can load the package with
\begin{verbatim}
  library(helloJavaWorld)
\end{verbatim}

and demonstrate it by simply calling
\begin{verbatim}
  > helloJavaWorld()
  [1] "Hello Java World!"
\end{verbatim}

\section{Acknowledgements}

We would like to thank Simon Urbanek for his continuous efforts
on the \texttt{rJava} package, without which it would not have
been possible to say Hello from Java to R. The following persons
provided feedback and suggestions that helped improve the
document: Duncan Murdoch, Hadley Wickham.

\end{document}
